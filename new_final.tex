%% LyX 2.4.2.1 created this file.  For more info, see https://www.lyx.org/.
%% Do not edit unless you really know what you are doing.
\documentclass[american]{article}
\usepackage[T1]{fontenc}
\usepackage[utf8]{inputenc}
\usepackage{amstext}
\usepackage{esint}
\usepackage{babel}
\begin{document}

\begin{itemize}
\item Firms with quantity $O=1$ to sell (supply inelastically because they are importers), choose a share $v$ (volume)
to move to $G$ (global/major market)
\item Firms differ in transportation costs (distance) per unit of transported
volume from major markets $\theta$ with density $f$
\item Choose local price $p_{l}(\theta)$ but global price determined by
equilibrium.

\begin{equation}
\int v(\theta)df(\theta) = Q^{G}(p)
\label{eq:global_equilibrium}
\end{equation}

\item Firms maximize profit by choosing share volume $v$ they move to global
market
\begin{equation}
\Pi = \max_{v,p_{l}} \left[ p_{l}(1-v) + v(p-\theta) \right]
\label{eq:profit_maximization}
\end{equation}

\item Local demand $Q^{l}(p_l)=Q_{l}p_{l}^{-\varepsilon_{Dl}}$ so $Q^{l}(p_l)/Q_{l}=p_{l}^{-\varepsilon_{Dl}}$
\item That is the equilibrium local price $p_{l}=(\frac{Q_{l}}{1-v})^{1/\varepsilon_{Dl}}$
\item So firms maximize

\begin{equation}
\max_{v} \left[ Q_{l}^{1/\varepsilon_{Dl}}(1-v)^{1-1/\varepsilon_{Dl}} + v(p-\theta) \right]
\label{eq:maximize_firm_profit}
\end{equation}

\item We get assuming $\varepsilon_{Dl}>1$: If $\theta>p$ then $v=0$
and firms don't want to send anything and profit is {[}Think of the
role of local elasticity{]}

\begin{equation}
\Pi = Q_{l}^{1/\varepsilon_{Dl}}
\label{eq:profit_no_shipping}
\end{equation}

\item If $\theta\leq p$, then $v$ according to, 
\begin{equation}
(1-1/\varepsilon_{Dl})Q_{l}^{1/\varepsilon_{Dl}}(1-v)^{-1/\varepsilon_{Dl}} = p - \theta
\label{eq:optimal_v}
\end{equation}

\item In summary firms with high transportation cost won't want to send
oil to the major market. So if the government puts important weight
on this major market, they might want to introduce oil subsidies.
\item Notice that if firms could set different prices in the hub, they could
just repercute the transportation cost to their price, and send some.
\item Now let's set a uniform oil tax $\tau$ and subsidy per unit
of moved volume $t(\theta)$

\item Firms maximize profit

\begin{equation}
\max_{v,p_{l}} \left[ p_{l}(1-\tau)(1-v) + v(p(1-\tau) + t(\theta) - \theta) \right]
\label{eq:max_profit_tax_subsidy}
\end{equation}

\item After the local price is set:
\begin{equation}
\max_{v} \left[ Q_{l}^{1/\varepsilon_{Dl}}(1-v)^{1-1/\varepsilon_{Dl}}(1-\tau) + v(p(1-\tau) + t(\theta) - \theta) \right]
\label{eq:max_profit_after_price_set}
\end{equation}

\item Assuming $\varepsilon > 1$: If $\theta > p(1-\tau_{g}) + t(\theta)$, then $v = 0$, and firms do not send anything. The profit is:

\begin{equation}
\Pi = Q_{l}^{1/\varepsilon_{Dl}}(1-\tau)
\label{eq:profit_no_shipping_tax}
\end{equation}

\item If $\theta \leq p(1-\tau_{g}) + t(\theta)$, then $v$ is determined by:
\begin{equation}
(1-1/\varepsilon_{Dl})Q_{l}^{1/\varepsilon_{Dl}}(1-v(\theta))^{-1/\varepsilon_{Dl}}(1-\tau) = p(1-\tau) + t(\theta) - \theta
\label{eq:optimal_v_tax}
\end{equation}

\item Government's problem maximize weighted consumer surplus
\begin{equation}
\hspace*{-2cm} % Adjust the value to move left
\max_{t(\theta), \tau_{g}} \left[
\mu^{G} \int_{p}^{\infty} Q^{G}(p') \, dp' + 
\mu^{L} \int \int_{p_{l}(\theta)} Q^{L}(p') \, dp' \, df(\theta) + 
\int \left[ \tau \cdot \left( p_{l}(\theta)(1-v(\theta)) + pv(\theta) \right) - t(\theta)v(\theta) \right] df(\theta)
\right]
\label{eq:objective_function}
\end{equation}



\noindent subject to:
\begin{equation}
\int v(\theta)df(\theta) = Q^{G}(p) = Q_{G}p^{-\varepsilon_{D_{G}}}
\quad \text{(global market clearing)}
\label{eq:global_equilibrium_constraint}
\end{equation}

\begin{equation}
Q^{L}(p_{l}(\theta)) = Q_{L}p_{l}(\theta)^{-\varepsilon_{Dl}} = 1-v(\theta) \quad \text{(local demand)}
\label{eq:local_demand_constraint}
\end{equation}

\begin{equation}
\hspace*{-2cm} % Adjust the value to move left
(1-1/\varepsilon_{Dl})Q_{l}^{1/\varepsilon_{Dl}}(1-v(\theta))^{-1/\varepsilon_{Dl}}(1-\tau) = p(1-\tau) + t(\theta) - \theta \quad 
\text{for } \theta \leq p(1-\tau_{g}) + t(\theta)
\quad \text{(price equality condition)}
\label{eq:optimal_v_with_tax_subsidy}
\end{equation}

\begin{equation}
\int \tau \left[ p_{l}(\theta)(1-v(\theta)) + pv(\theta) \right] df(\theta) = \int t(\theta)v(\theta)df(\theta) 
\quad \text{(government budget constraint)}
\label{eq:government_budget_constraint}
\end{equation}

\item Will optimize this to find the optimal subsidy schedule.
\item Let's have linear subsidy $t(\theta)=t\theta$
\item Send cutoff is then $\theta\leq\frac{p(1-\tau_{g})}{1-t}$, denote
$\tilde{\theta}$ the minimum cut-off resulting.
\item The choice of shipped volume below $\tilde{\theta}$ is
\begin{equation}
(1-1/\varepsilon_{Dl})Q_{l}^{1/\varepsilon_{Dl}}(1-v(\theta))^{-1/\varepsilon_{Dl}}\frac{\tilde{\theta}}{p} = \tilde{\theta} - \theta \quad \text{for } \theta \leq \tilde{\theta}
\label{eq:shipped_volume}
\end{equation}

By rearranging the equality we obtain:




\begin{equation}
(1-v(\theta)) = \left(\frac{p(\tilde{\theta}-\theta)}{\left(1-\frac{1}{\varepsilon_{Dl}}\right)Q_{l}^{1/\varepsilon_{Dl}}\tilde{\theta}}\right)^{-\varepsilon_{Dl}} = \left(\frac{\left(1-\frac{1}{\varepsilon_{Dl}}\right)Q_{l}^{1/\varepsilon_{Dl}}\tilde{\theta}}{p(\tilde{\theta}-\theta)}\right)^{\varepsilon_{Dl}} = Q_{l} \cdot \left(\frac{\left(1-\frac{1}{\varepsilon_{Dl}}\right)\tilde{\theta}}{p(\tilde{\theta}-\theta)}\right)^{\varepsilon_{Dl}} 
\label{eq: share_total_prod_local}
\end{equation} 




and then:

\begin{equation}
(1-v(\theta))^{\frac{\varepsilon_{Dl}-1}{\varepsilon_{Dl}}} = \left(\frac{p(\tilde{\theta}-\theta)}{\left(1-\frac{1}{\varepsilon_{Dl}}\right)Q_{l}^{1/\varepsilon_{Dl}}\tilde{\theta}}\right)^{-(\varepsilon_{Dl}-1)} = \left(\frac{\left(1-\frac{1}{\varepsilon_{Dl}}\right)Q_{l}^{1/\varepsilon_{Dl}}\tilde{\theta}}{p(\tilde{\theta}-\theta)}\right)^{(\varepsilon_{Dl}-1)} 
\label{eq: share_total_prod_local}
\end{equation} 

\begin{equation}
(1-v(\theta))^{\frac{\varepsilon_{Dl}-1}{\varepsilon_{Dl}}} = Q_{l}^{\frac{\varepsilon_{Dl}-1}{\varepsilon_{Dl}}} \left(\frac{\left(1-\frac{1}{\varepsilon_{Dl}}\right)\tilde{\theta}}{p(\tilde{\theta}-\theta)}\right)^{(\varepsilon_{Dl}-1)}
\label{eq: share_total_prod_local_simplified}
\end{equation} 



\item The objective:
\begin{equation}
CS_{L}(\theta) = \int_{p_{l}(\theta)}^{\infty} Q^{L}(p) \, dp = \int_{p_{l}(\theta)}^{\infty} Q_{L} \cdot p^{-\varepsilon_{Dl}} \, dp = \frac{Q_{L}}{\varepsilon_{Dl}-1} \cdot p_{l}(\theta)^{-(\varepsilon_{Dl}-1)} = \frac{Q_{L}}{\varepsilon_{Dl}-1} \cdot p_{l}(\theta)^{1-\varepsilon_{Dl}}
\label{eq:consumer_surplus_L_initial}
\end{equation}

\item Developing it further:
\begin{equation}
CS_{L}(\theta) = \frac{Q_{L}}{\varepsilon_{Dl}-1} \cdot \left(\frac{1-v}{Q_{L}}\right)^{\frac{\varepsilon_{Dl}-1}{\varepsilon_{Dl}}} = \frac{Q_{L}^{\frac{1}{\varepsilon_{Dl}}}}{\varepsilon_{Dl}-1} \cdot (1-v)^{\frac{\varepsilon_{Dl}-1}{\varepsilon_{Dl}}}
\label{eq:consumer_surplus_L_final}
\end{equation}


\begin{equation}
CS_{L}(\theta) = \frac{Q_{L}^{\frac{1}{\varepsilon_{Dl}}}}{\varepsilon_{Dl}-1} \cdot \left(\frac{\left(1-\frac{1}{\varepsilon_{Dl}}\right)Q_{l}^{1/\varepsilon_{Dl}}\tilde{\theta}}{p(\tilde{\theta}-\theta)}\right)^{(\varepsilon_{Dl}-1)} = \frac{Q_{L}}{\varepsilon_{Dl}-1} \cdot \left(\frac{\left(1-\frac{1}{\varepsilon_{Dl}}\right)\tilde{\theta}}{p(\tilde{\theta}-\theta)}\right)^{(\varepsilon_{Dl}-1)}
\label{eq:consumer_surplus_L_theta}
\end{equation}

\begin{equation}
CS_{G} = \int_{p}^{\infty} Q^{G}(p) \, dp = \int_{p}^{\infty} Q_{G} \cdot p^{-\varepsilon_{DG}} \, dp = \frac{Q_{G}}{\varepsilon_{DG}-1} \cdot p^{-(\varepsilon_{DG}-1)} = \frac{Q_{G}}{\varepsilon_{DG}-1} \cdot p^{1-\varepsilon_{DG}}
\label{eq:consumer_surplus_G}
\end{equation}

\item Budget constraint

\begin{equation}
\int \tau \left[ p_{l}(\theta)(1-v(\theta)) + pv(\theta) \right] df(\theta) = t \int \theta v(\theta) df(\theta)
\label{eq:budget_constraint1}
\end{equation}

\begin{equation}
\int \tau p_{l}(\theta)(1-v(\theta)) df(\theta) + \tau Q_{G}p^{1-\varepsilon_{D_{G}}} = t \int \theta v(\theta) df(\theta)
\label{eq:budget_constraint2}
\end{equation}

\begin{equation}
\tau p_{l}(1-v) = \tau Q_{L} \left( 1-\frac{1}{\varepsilon_{Dl}} \right)^{\varepsilon_{Dl}-1} \left( \frac{\tilde{\theta}}{p(\tilde{\theta}-\theta)} \right)^{\varepsilon_{Dl}-1}
\label{eq:price_tax_relationship}
\end{equation}

\begin{equation}
tv(\theta)\theta = t\theta \left[ 1 - \left(\frac{\left(1-\frac{1}{\varepsilon_{Dl}}\right)Q_{l}^{1/\varepsilon_{Dl}}\tilde{\theta}}{p(\tilde{\theta}-\theta)}\right)^{\varepsilon_{Dl}} \right] = t\theta \left[ 1 - Q_{l} \cdot \left(\frac{\left(1-\frac{1}{\varepsilon_{Dl}}\right)\tilde{\theta}}{p(\tilde{\theta}-\theta)}\right)^{\varepsilon_{Dl}} \right]
\label{eq:subsidy_equation}
\end{equation}

\begin{equation}
\hspace*{-2cm} % Adjust the value to move left
\tau Q_{L} \left( 1-\frac{1}{\varepsilon_{Dl}} \right)^{\varepsilon_{Dl}-1} \int \left( \frac{\tilde{\theta}}{p(\tilde{\theta}-\theta)} \right)^{\varepsilon_{Dl}-1} df(\theta) + \tau Q_{G}p^{1-\varepsilon_{D_{G}}} =
t \int \theta \left[ 1 - Q_{l} \cdot \left(\frac{\left(1-\frac{1}{\varepsilon_{Dl}}\right)\tilde{\theta}}{p(\tilde{\theta}-\theta)}\right)^{\varepsilon_{Dl}} \right] df(\theta)
\label{eq:integral_constraint}
\end{equation}

\item Maximization of weighted surplus:
\begin{equation}
\max_{t, \tau, p} \left[
\mu^{G}\frac{Q_{G}}{\varepsilon_{DG}-1} \cdot p^{1-\varepsilon_{DG}} + 
\mu^{L} \int_{0}^{\tilde{\theta}} \frac{Q_{L}}{\varepsilon_{Dl}-1} \cdot \left(\frac{\left(1-\frac{1}{\varepsilon_{Dl}}\right)\tilde{\theta}}{p(\tilde{\theta}-\theta)}\right)^{(\varepsilon_{Dl}-1)} df(\theta)
\right]
\label{eq:max_weighted_surplus}
\end{equation}


\item  subject to the budget constraint:
\begin{equation}
\hspace*{-2cm} % Adjust the value to move left
\tau Q_{L} \left( 1-\frac{1}{\varepsilon_{Dl}} \right)^{\varepsilon_{Dl}-1} \int \left( \frac{\tilde{\theta}}{p(\tilde{\theta}-\theta)} \right)^{\varepsilon_{Dl}-1} df(\theta) + \tau Q_{G}p^{1-\varepsilon_{D_{G}}} =
t \int \theta \left[ 1 - Q_{l} \cdot \left(\frac{\left(1-\frac{1}{\varepsilon_{Dl}}\right)\tilde{\theta}}{p(\tilde{\theta}-\theta)}\right)^{\varepsilon_{Dl}} \right] df(\theta)
\label{eq:budget_constraint_3}
\end{equation}




\item global market clearing
\begin{equation}
\int_{0}^{\tilde{\theta}}\left[ 1 - Q_{l} \cdot \left(\frac{\left(1-\frac{1}{\varepsilon_{Dl}}\right)\tilde{\theta}}{p(\tilde{\theta}-\theta)}\right)^{\varepsilon_{Dl}} \right] df(\theta) = Q_{G}p^{-\varepsilon_{D_{G}}}
\label{eq:market_clearing_condition}
\end{equation}

\item Cutoff:
\begin{equation}
\tilde{\theta} = \frac{p(1-\tau_{g})}{1-t}
\label{eq:cutoff_condition}
\end{equation}

\end{itemize}

Assuming \( f(\theta) = 1 \) , we can rewrite the objective function and the constraints. The integral on the left hand side of the market clearing condition can then be easily computed:

\[
\int_{0}^{\tilde{\theta}} \left[ 1 - Q_{l} \cdot \left(\frac{\left(1-\frac{1}{\varepsilon_{Dl}}\right)\tilde{\theta}}{p(\tilde{\theta}-\theta)}\right)^{\varepsilon_{Dl}} \right] d\theta
\]

The first term becomes:

\[
 \int_{0}^{\tilde{\theta}} 1 \, d\theta = \tilde{\theta}
\]

For the second term, we have:

\[
 \int_{0}^{\tilde{\theta}} Q_{l} \cdot \left(\frac{\left(1-\frac{1}{\varepsilon_{Dl}}\right)\tilde{\theta}}{p(\tilde{\theta}-\theta)}\right)^{\varepsilon_{Dl}} d\theta = Q_{l} \cdot \left( \frac{\left( 1-\frac{1}{\varepsilon_{Dl}} \right)  \tilde{\theta}}{p} \right)^{\varepsilon_{Dl}} \int_{0}^{\tilde{\theta}} \frac{1}{(\tilde{\theta} - \theta)^{\varepsilon_{Dl}}} \, d\theta
\] 


Using the substitution \( u = \tilde{\theta} - \theta \) we get:


\begin{equation}
\int_{0}^{\tilde{\theta}} \frac{1}{(\tilde{\theta} - \theta)^{\varepsilon_{Dl}}} \, d\theta = \int_{\tilde{\theta}}^{0} u^{-\varepsilon_{Dl}} (-du) =  \int_{0}^{\tilde{\theta}} u^{-\varepsilon_{Dl}} \, du = \frac{u^{1-\varepsilon_{Dl}}}{1-\varepsilon_{Dl}} \Big|_{0}^{\tilde{\theta}} = \frac{\tilde{\theta}^{1-\varepsilon_{Dl}}}{1-\varepsilon_{Dl}}
\label{eq:integral_substitution_market_clearing}
\end{equation}

The second term becomes:

\[
 Q_{l} \cdot \left( \frac{\left( 1-\frac{1}{\varepsilon_{Dl}} \right)  \tilde{\theta}}{p} \right)^{\varepsilon_{Dl}} \cdot \frac{\tilde{\theta}^{1-\varepsilon_{Dl}}}{1-\varepsilon_{Dl}}.
\]

Simplifying further, we get:

\[
 Q_{l} \cdot \frac{\left( 1-\frac{1}{\varepsilon_{Dl}} \right)^{\varepsilon_{Dl}}  \cdot \tilde{\theta} }{p^{\varepsilon_{Dl}} (1-\varepsilon_{Dl})}
\]

Combining the first and second terms, we get:

\[
\tilde{\theta} -  Q_{l} \cdot \frac{\left( 1-\frac{1}{\varepsilon_{Dl}} \right)^{\varepsilon_{Dl}}  \cdot \tilde{\theta} }{p^{\varepsilon_{Dl}} (1-\varepsilon_{Dl})} = \tilde{\theta} + Q_{l} \cdot \frac{\left( 1-\frac{1}{\varepsilon_{Dl}} \right)^{\varepsilon_{Dl}}  \cdot \tilde{\theta} }{p^{\varepsilon_{Dl}} (\varepsilon_{Dl}-1)}
\]

The market clearing condition can then be rewritten as:


\begin{equation}
\tilde{\theta} + Q_{l} \cdot \frac{\left( 1-\frac{1}{\varepsilon_{Dl}} \right)^{\varepsilon_{Dl}}  \cdot \tilde{\theta} }{p^{\varepsilon_{Dl}} (\varepsilon_{Dl}-1)} =  Q_{G}p^{-\varepsilon_{D_{G}}}
\label{eq:market_clearing_condition_simplified}
\end{equation}\\



 Now we will consider the budget constraint (\ref{eq:budget_constraint_3}). By using the same substitution technique as in \ref{eq:integral_substitution_market_clearing}, the integral on the left hand side of the equation can be written as::

\begin{equation}
\int_{0}^{\tilde{\theta}} \frac{1}{(\tilde{\theta} - \theta)^{\varepsilon_{Dl}-1}} \, d\theta = \int_{\tilde{\theta}}^{0} u^{-(\varepsilon_{Dl}-1)} (-du) =  \int_{0}^{\tilde{\theta}} u^{-(\varepsilon_{Dl}-1)} \, du = \frac{u^{2-\varepsilon_{Dl}}}{2-\varepsilon_{Dl}} \Big|_{0}^{\tilde{\theta}} = \frac{\tilde{\theta}^{2-\varepsilon_{Dl}}}{2-\varepsilon_{Dl}}
\label{eq:integral_substitution_budget_constraint}
\end{equation}\\
 

The left-hand side of the equation of the budget constraint can then be simplified  as:
\[
\tau Q_{L} \left( 1-\frac{1}{\varepsilon_{Dl}} \right)^{\varepsilon_{Dl}-1} \cdot \frac{\tilde{\theta}^{\varepsilon_{Dl}-1}}{p^{\varepsilon_{Dl}-1}} \cdot \frac{\tilde{\theta}^{2-\varepsilon_{Dl}}}{2-\varepsilon_{Dl}} + \tau Q_{G}p^{1-\varepsilon_{D_{G}}} =  \frac{\tau Q_{L} \cdot \theta }{p^{\varepsilon_{Dl}-1}\cdot (2-\varepsilon_{Dl})}  \left( 1-\frac{1}{\varepsilon_{Dl}} \right)^{\varepsilon_{Dl}-1} + \tau Q_{G}p^{1-\varepsilon_{D_{G}}}
\]\\


Now we will focus on the right hand side: 

We are still assuming \( f(\theta) = 1 \) (uniform distribution). The term on the right-hand side of the budget constraint can then be computed:

\[
t \int \theta \left[ 1 - Q_{l} \cdot \left(\frac{\left(1-\frac{1}{\varepsilon_{Dl}}\right)\tilde{\theta}}{p(\tilde{\theta}-\theta)}\right)^{\varepsilon_{Dl}} \right]  \, d\theta.
\]

The first term becomes:

\[
t \int_{0}^{\tilde{\theta}} \theta \, d\theta = t  \cdot \frac{\theta^2}{2} \Big|_{0}^{\tilde{\theta}} = t \cdot \frac{\tilde{\theta}^2}{2}.
\]

For the second term, we have:

\[
t  \int_{0}^{\tilde{\theta}} \theta  \cdot Q_{l} \cdot \left(\frac{\left(1-\frac{1}{\varepsilon_{Dl}}\right)\tilde{\theta}}{p(\tilde{\theta}-\theta)}\right)^{\varepsilon_{Dl}} \, d\theta = t   \cdot Q_{l} \cdot \left(\frac{\left(1-\frac{1}{\varepsilon_{Dl}}\right)\tilde{\theta}}{p}\right)^{\varepsilon_{Dl}} \cdot  \int_{0}^{\tilde{\theta}} \frac{\theta}{(\tilde{\theta} - \theta)^{\varepsilon_{Dl}}} \, d\theta.
\]

Using the substitution \( u = \tilde{\theta} - \theta \), we have:

\begin{equation}
\int_{0}^{\tilde{\theta}} \frac{\theta}{(\tilde{\theta} - \theta)^{\varepsilon_{Dl}}} \, d\theta = \int_{\tilde{\theta}}^{0} \frac{\tilde{\theta} - u}{u^{\varepsilon_{Dl}}} (-du) = \int_{0}^{\tilde{\theta}} \frac{\tilde{\theta}}{u^{\varepsilon_{Dl}}} \, du - \int_{0}^{\tilde{\theta}} u^{-(\varepsilon_{Dl} - 1)} \, du.
\label{eq:split_integral}
\end{equation}

Each term is computed as follows:
For the first term:
\[
\int_{0}^{\tilde{\theta}} \frac{\tilde{\theta}}{u^{\varepsilon_{Dl}}} \, du = \tilde{\theta} \int_{0}^{\tilde{\theta}} u^{-\varepsilon_{Dl}} \, du = \tilde{\theta} \cdot \frac{u^{1 - \varepsilon_{Dl}}}{1 - \varepsilon_{Dl}} \Big|_{0}^{\tilde{\theta}} = \frac{\tilde{\theta} \cdot \tilde{\theta}^{1 - \varepsilon_{Dl}}}{1 - \varepsilon_{Dl}} = \frac{\tilde{\theta}^{2 - \varepsilon_{Dl}}}{1 - \varepsilon_{Dl}}.
\]

For the second term:
\[
\int_{0}^{\tilde{\theta}} u^{-(\varepsilon_{Dl} - 1)} \, du = \frac{u^{2 - \varepsilon_{Dl}}}{2 - \varepsilon_{Dl}} \Big|_{0}^{\tilde{\theta}} = \frac{\tilde{\theta}^{2 - \varepsilon_{Dl}}}{2 - \varepsilon_{Dl}}.
\]

Combining the two terms, we get:

\[
\int_{0}^{\tilde{\theta}} \frac{\theta}{(\tilde{\theta} - \theta)^{\varepsilon_{Dl}}} \, d\theta = \frac{\tilde{\theta}^{2 - \varepsilon_{Dl}}}{1 - \varepsilon_{Dl}} - \frac{\tilde{\theta}^{2 - \varepsilon_{Dl}}}{2 - \varepsilon_{Dl}} = \frac{\tilde{\theta}^{2 - \varepsilon_{Dl}}}{(1 - \varepsilon_{Dl})(2 - \varepsilon_{Dl})}
\]

The second term becomes:

\[
t   \cdot Q_{l} \cdot \left(\frac{\left(1-\frac{1}{\varepsilon_{Dl}}\right)\tilde{\theta}}{p}\right)^{\varepsilon_{Dl}} \cdot   \frac{\tilde{\theta}^{2 - \varepsilon_{Dl}}}{(1 - \varepsilon_{Dl})(2 - \varepsilon_{Dl})}
\]

Simplifying further, we get:

\[
t \cdot \tilde{\theta}^2 \cdot \frac{\left( 1 - \frac{1}{\varepsilon_{Dl}} \right)^{\varepsilon_{Dl}} Q_L}{p^{\varepsilon_{Dl}} (1 - \varepsilon_{Dl})(2 - \varepsilon_{Dl})}
\]

Combining the first and second terms, we have:

\[
t \cdot \tilde{\theta}^2 \cdot\left[\frac{1}{2} -   \frac{\left( 1 - \frac{1}{\varepsilon_{Dl}} \right)^{\varepsilon_{Dl}} Q_L}{p^{\varepsilon_{Dl}} (1 - \varepsilon_{Dl})(2 - \varepsilon_{Dl})}) \right]
\]

The budget constraint can then be expressed as:

\[
 \frac{\tau Q_{L} \cdot \theta }{p^{\varepsilon_{Dl}-1}\cdot (2-\varepsilon_{Dl})}  \left( 1-\frac{1}{\varepsilon_{Dl}} \right)^{\varepsilon_{Dl}-1} + \tau Q_{G}p^{1-\varepsilon_{D_{G}}} = t \cdot \tilde{\theta}^2 \cdot\left[\frac{1}{2} -   \frac{\left( 1 - \frac{1}{\varepsilon_{Dl}} \right)^{\varepsilon_{Dl}} Q_L}{p^{\varepsilon_{Dl}} (1 - \varepsilon_{Dl})(2 - \varepsilon_{Dl})}) \right]
\]

Replacing $\tilde{\theta}$, we get:



\begin{equation}
\hspace*{-2cm} 
\frac{p(1-\tau_{g})}{1-t} \cdot \frac{\tau Q_{L}}{p^{\varepsilon_{Dl}-1} (2-\varepsilon_{Dl})}\left( 1-\frac{1}{\varepsilon_{Dl}} \right)^{\varepsilon_{Dl}-1} + 
\tau Q_{G}p^{1-\varepsilon_{D_{G}}} = t  \cdot \left[\ \frac{p(1-\tau_{g})}{1-t} \right]^2 \cdot\left[\frac{1}{2} -   \frac{\left( 1 - \frac{1}{\varepsilon_{Dl}} \right)^{\varepsilon_{Dl}} Q_L}{p^{\varepsilon_{Dl}} (1 - \varepsilon_{Dl})(2 - \varepsilon_{Dl})}) \right]
\label{eq:budget_constraint_simplified}
\end{equation}





Now, doing the same substitution as \ref{eq:integral_substitution_budget_constraint}, we can simplify the objective function (\ref{eq:max_weighted_surplus}):





\[
\max_{t, \tau, p} \left[
\mu^{G} \frac{Q_{G}}{\varepsilon_{DG}-1} \cdot p^{-(\varepsilon_{DG}-1)} + 
\mu^{L} \frac{Q_{L}}{\varepsilon_{DL} - 1} \cdot \left( \frac{\left( 1 - \frac{1}{\varepsilon_{DL}} \right) \tilde{\theta}}{p} \right)^{\varepsilon_{DL} - 1} \cdot \frac{\tilde{\theta}^{2 - \varepsilon_{DL}}}{2 - \varepsilon_{DL}}
\right]
\]

By further simplifying we get:

\[
\max_{t, \tau, p} \left[
\mu^{G} \frac{Q_{G}}{(\varepsilon_{DG}-1) } \cdot p^{-(\varepsilon_{DG}-1)} + 
\mu^{L} \frac{Q_{L} \cdot \theta}{(\varepsilon_{DL} - 1) \cdot (2 - \varepsilon_{DL})} \cdot \left( \frac{1 - \frac{1}{\varepsilon_{DL}}  }{p} \right)^{\varepsilon_{DL} - 1} 
\right]
\]

Replacing $\tilde{\theta}$, we get:


\begin{equation}
\max_{t, \tau, p} \left[
\mu^{G} \frac{Q_{G}}{(\varepsilon_{DG}-1) } \cdot p^{-(\varepsilon_{DG}-1)} + 
\mu^{L} \frac{Q_{L} \cdot \frac{p(1-\tau_{g})}{1-t}}{(\varepsilon_{DL} - 1) \cdot (2 - \varepsilon_{DL})} \cdot \left( \frac{1 - \frac{1}{\varepsilon_{DL}}  }{p} \right)^{\varepsilon_{DL} - 1} 
\right]
\label{eq:max_weighted_surplus_simplified}
\end{equation}\\




Finally by combining everything toghether, we get the following simplified version of the problem:

\begin{itemize}
    \item Maximization of weighted surplus:
    \begin{equation}
\max_{t, \tau, p} \left[
\mu^{G} \frac{Q_{G}}{(\varepsilon_{DG}-1) } \cdot p^{-(\varepsilon_{DG}-1)} + 
\mu^{L} \frac{Q_{L} \cdot \frac{p(1-\tau_{g})}{1-t}}{(\varepsilon_{DL} - 1) \cdot (2 - \varepsilon_{DL})} \cdot \left( \frac{1 - \frac{1}{\varepsilon_{DL}}  }{p} \right)^{\varepsilon_{DL} - 1} 
\right]
    \label{eq:max_weighted_surplus_simplified_model}
    \end{equation}
    
    \item Subject to the budget constraint:
    \begin{equation}
\hspace*{-2cm} 
\frac{p(1-\tau_{g})}{1-t} \cdot \frac{\tau Q_{L}}{p^{\varepsilon_{Dl}-1} (2-\varepsilon_{Dl})}\left( 1-\frac{1}{\varepsilon_{Dl}} \right)^{\varepsilon_{Dl}-1} + 
\tau Q_{G}p^{1-\varepsilon_{D_{G}}} = t  \cdot \left[\ \frac{p(1-\tau_{g})}{1-t} \right]^2 \cdot\left[\frac{1}{2} -   \frac{\left( 1 - \frac{1}{\varepsilon_{Dl}} \right)^{\varepsilon_{Dl}} Q_L}{p^{\varepsilon_{Dl}} (1 - \varepsilon_{Dl})(2 - \varepsilon_{Dl})}) \right]
    \label{eq:budget_constraint_simplified_model}
    \end{equation}
    
    \item Global market clearing:
    \begin{equation}
\frac{p(1-\tau_{g})}{1-t} \cdot \left[\ 1+ \frac{\left( 1-\frac{1}{\varepsilon_{Dl}} \right)^{\varepsilon_{Dl}} Q_{L}  }{p^{\varepsilon_{Dl}} (\varepsilon_{Dl}-1)} \right]=  Q_{G}p^{-\varepsilon_{D_{G}}}
    \label{eq:market_clearing_condition_simplified_model}
    \end{equation}
\end{itemize}





Now we want to solve the problem



\end{document}
